% Options for packages loaded elsewhere
\PassOptionsToPackage{unicode}{hyperref}
\PassOptionsToPackage{hyphens}{url}
%
\documentclass[
  ngerman,
  ignorenonframetext,
]{beamer}
\usepackage{pgfpages}
\setbeamertemplate{caption}[numbered]
\setbeamertemplate{caption label separator}{: }
\setbeamercolor{caption name}{fg=normal text.fg}
\beamertemplatenavigationsymbolsempty
% Prevent slide breaks in the middle of a paragraph
\widowpenalties 1 10000
\raggedbottom
\setbeamertemplate{part page}{
  \centering
  \begin{beamercolorbox}[sep=16pt,center]{part title}
    \usebeamerfont{part title}\insertpart\par
  \end{beamercolorbox}
}
\setbeamertemplate{section page}{
  \centering
  \begin{beamercolorbox}[sep=12pt,center]{part title}
    \usebeamerfont{section title}\insertsection\par
  \end{beamercolorbox}
}
\setbeamertemplate{subsection page}{
  \centering
  \begin{beamercolorbox}[sep=8pt,center]{part title}
    \usebeamerfont{subsection title}\insertsubsection\par
  \end{beamercolorbox}
}
\AtBeginPart{
  \frame{\partpage}
}
\AtBeginSection{
  \ifbibliography
  \else
    \frame{\sectionpage}
  \fi
}
\AtBeginSubsection{
  \frame{\subsectionpage}
}
\usepackage{amsmath,amssymb}
\usepackage{lmodern}
\usepackage{iftex}
\ifPDFTeX
  \usepackage[T1]{fontenc}
  \usepackage[utf8]{inputenc}
  \usepackage{textcomp} % provide euro and other symbols
\else % if luatex or xetex
  \usepackage{unicode-math}
  \defaultfontfeatures{Scale=MatchLowercase}
  \defaultfontfeatures[\rmfamily]{Ligatures=TeX,Scale=1}
\fi
\usetheme[]{Berkeley}
% Use upquote if available, for straight quotes in verbatim environments
\IfFileExists{upquote.sty}{\usepackage{upquote}}{}
\IfFileExists{microtype.sty}{% use microtype if available
  \usepackage[]{microtype}
  \UseMicrotypeSet[protrusion]{basicmath} % disable protrusion for tt fonts
}{}
\makeatletter
\@ifundefined{KOMAClassName}{% if non-KOMA class
  \IfFileExists{parskip.sty}{%
    \usepackage{parskip}
  }{% else
    \setlength{\parindent}{0pt}
    \setlength{\parskip}{6pt plus 2pt minus 1pt}}
}{% if KOMA class
  \KOMAoptions{parskip=half}}
\makeatother
\usepackage{xcolor}
\IfFileExists{xurl.sty}{\usepackage{xurl}}{} % add URL line breaks if available
\IfFileExists{bookmark.sty}{\usepackage{bookmark}}{\usepackage{hyperref}}
\hypersetup{
  pdftitle={Thema 1: Was ist Inferenzstatistik?},
  pdfauthor={Prof.~Sauer},
  pdflang={de-DE},
  hidelinks,
  pdfcreator={LaTeX via pandoc}}
\urlstyle{same} % disable monospaced font for URLs
\newif\ifbibliography
\setlength{\emergencystretch}{3em} % prevent overfull lines
\providecommand{\tightlist}{%
  \setlength{\itemsep}{0pt}\setlength{\parskip}{0pt}}
\setcounter{secnumdepth}{-\maxdimen} % remove section numbering
%\setbeamertemplate{page number in head/foot}[totalframenumber]
\setbeamertemplate{footline}[frame number]
\ifXeTeX
  % Load polyglossia as late as possible: uses bidi with RTL langages (e.g. Hebrew, Arabic)
  \usepackage{polyglossia}
  \setmainlanguage[]{german}
\else
  \usepackage[main=ngerman]{babel}
% get rid of language-specific shorthands (see #6817):
\let\LanguageShortHands\languageshorthands
\def\languageshorthands#1{}
\fi
\ifLuaTeX
  \usepackage{selnolig}  % disable illegal ligatures
\fi

\title{Thema 1: Was ist Inferenzstatistik?}
\subtitle{QM2, ROS, Kap. 1, ReThink\_v1, Kap. 1}
\author{Prof.~Sauer}
\date{WiSe 21}
\institute{AWM, HS Ansbach}

\begin{document}
\frame{\titlepage}

\begin{frame}[allowframebreaks]
  \tableofcontents[hideallsubsections]
\end{frame}
\begin{frame}{Was ist Inferenzstatistik?}
\protect\hypertarget{was-ist-inferenzstatistik}{}
\hyphenation{Wahr-schein-lichk-keit}

\begin{block}{Deskriptiv- vs.~Inferenzstatistik}
\protect\hypertarget{deskriptiv--vs.-inferenzstatistik}{}
\tikzset{
zahl/.style={fill=blue!70!yellow, text=black, label=center:\textsf{\Large Z}},
kopf/.style={fill=orange!90!blue, label=center:\textsf{\Large K}}
}

\begin{tikzpicture}
[scale=1.5, transform shape,
thick,
every node/.style={draw, circle, minimum size=10mm},
grow=down, %Zeichenrichtung
level 1/.style={sibling distance=3cm},
level 2/.style={sibling distance=4cm},
level 3/.style={sibling distance=2cm},
level distance=1.25cm]

\node[fill=gray!40, shape=rectangle, rounded corners, minimum width = 6cm] {Münzwurf}
child{ node[shape=circle split,draw,line width=1pt,minimum size=10mm,inner sep=0mm, font=\sffamily\large, rotate=30] (Start) { \rotatebox{-30}{K} \nodepart{lower} \rotatebox{-30}{Z}}
child {node[kopf] (A) {}
child {node[kopf] (B) {}}
child {node[zahl] (C) {}}
}
child {node[zahl] (D) {}
child {node[kopf] (E) {}}
child {node[zahl] (F) {}}
}
};

%Füllung der Wurzel = "Start"
\begin{scope}[on background layer, rotate=30]
\fill[kopf] (Start.base) ([xshift=0mm]Start.east) arc (0:180:5mm)--cycle;
\fill[zahl] (Start.base) ([xshift= 0pt]Start.west) arc (180:360:5mm)--cycle;
\end{scope}

%Beschriftung
\path (Start) -- (A) node [draw=none, near start, left] {$0.5$};
\path (A) -- (B) node [draw=none, near start, left] {$0.5$};
\path (A) -- (C) node [draw=none, near start, right] {$0.5$};
\path (Start) -- (D) node [draw=none, near start, right] {$0.5$};
\path (D) -- (E) node [draw=none, near start, left] {$0.5$};
\path (D) -- (F) node [draw=none, near start, right] {$0.5$};
%
\node[below=11pt, draw=none, name=X] at (B) {$0.25$};
\node[below=11pt, draw=none] at (C) {$0.25$};
\node[below=11pt, draw=none, name=Y] at (E) {$0.25$};
\node[below=11pt, draw=none] at (F) {$0.25$};
%
\draw[densely dashed, rounded corners, thin] (X.south west) rectangle (Y.north east);

\end{tikzpicture}

\%===========
\end{block}
\end{frame}

\end{document}
